% Options for packages loaded elsewhere
\PassOptionsToPackage{unicode}{hyperref}
\PassOptionsToPackage{hyphens}{url}
\PassOptionsToPackage{dvipsnames,svgnames,x11names}{xcolor}
%
\documentclass[
  letterpaper,
  DIV=11,
  numbers=noendperiod]{scrartcl}

\usepackage{amsmath,amssymb}
\usepackage{iftex}
\ifPDFTeX
  \usepackage[T1]{fontenc}
  \usepackage[utf8]{inputenc}
  \usepackage{textcomp} % provide euro and other symbols
\else % if luatex or xetex
  \usepackage{unicode-math}
  \defaultfontfeatures{Scale=MatchLowercase}
  \defaultfontfeatures[\rmfamily]{Ligatures=TeX,Scale=1}
\fi
\usepackage{lmodern}
\ifPDFTeX\else  
    % xetex/luatex font selection
\fi
% Use upquote if available, for straight quotes in verbatim environments
\IfFileExists{upquote.sty}{\usepackage{upquote}}{}
\IfFileExists{microtype.sty}{% use microtype if available
  \usepackage[]{microtype}
  \UseMicrotypeSet[protrusion]{basicmath} % disable protrusion for tt fonts
}{}
\makeatletter
\@ifundefined{KOMAClassName}{% if non-KOMA class
  \IfFileExists{parskip.sty}{%
    \usepackage{parskip}
  }{% else
    \setlength{\parindent}{0pt}
    \setlength{\parskip}{6pt plus 2pt minus 1pt}}
}{% if KOMA class
  \KOMAoptions{parskip=half}}
\makeatother
\usepackage{xcolor}
\setlength{\emergencystretch}{3em} % prevent overfull lines
\setcounter{secnumdepth}{-\maxdimen} % remove section numbering
% Make \paragraph and \subparagraph free-standing
\ifx\paragraph\undefined\else
  \let\oldparagraph\paragraph
  \renewcommand{\paragraph}[1]{\oldparagraph{#1}\mbox{}}
\fi
\ifx\subparagraph\undefined\else
  \let\oldsubparagraph\subparagraph
  \renewcommand{\subparagraph}[1]{\oldsubparagraph{#1}\mbox{}}
\fi


\providecommand{\tightlist}{%
  \setlength{\itemsep}{0pt}\setlength{\parskip}{0pt}}\usepackage{longtable,booktabs,array}
\usepackage{calc} % for calculating minipage widths
% Correct order of tables after \paragraph or \subparagraph
\usepackage{etoolbox}
\makeatletter
\patchcmd\longtable{\par}{\if@noskipsec\mbox{}\fi\par}{}{}
\makeatother
% Allow footnotes in longtable head/foot
\IfFileExists{footnotehyper.sty}{\usepackage{footnotehyper}}{\usepackage{footnote}}
\makesavenoteenv{longtable}
\usepackage{graphicx}
\makeatletter
\def\maxwidth{\ifdim\Gin@nat@width>\linewidth\linewidth\else\Gin@nat@width\fi}
\def\maxheight{\ifdim\Gin@nat@height>\textheight\textheight\else\Gin@nat@height\fi}
\makeatother
% Scale images if necessary, so that they will not overflow the page
% margins by default, and it is still possible to overwrite the defaults
% using explicit options in \includegraphics[width, height, ...]{}
\setkeys{Gin}{width=\maxwidth,height=\maxheight,keepaspectratio}
% Set default figure placement to htbp
\makeatletter
\def\fps@figure{htbp}
\makeatother

\KOMAoption{captions}{tableheading}
\makeatletter
\@ifpackageloaded{caption}{}{\usepackage{caption}}
\AtBeginDocument{%
\ifdefined\contentsname
  \renewcommand*\contentsname{Table of contents}
\else
  \newcommand\contentsname{Table of contents}
\fi
\ifdefined\listfigurename
  \renewcommand*\listfigurename{List of Figures}
\else
  \newcommand\listfigurename{List of Figures}
\fi
\ifdefined\listtablename
  \renewcommand*\listtablename{List of Tables}
\else
  \newcommand\listtablename{List of Tables}
\fi
\ifdefined\figurename
  \renewcommand*\figurename{Figure}
\else
  \newcommand\figurename{Figure}
\fi
\ifdefined\tablename
  \renewcommand*\tablename{Table}
\else
  \newcommand\tablename{Table}
\fi
}
\@ifpackageloaded{float}{}{\usepackage{float}}
\floatstyle{ruled}
\@ifundefined{c@chapter}{\newfloat{codelisting}{h}{lop}}{\newfloat{codelisting}{h}{lop}[chapter]}
\floatname{codelisting}{Listing}
\newcommand*\listoflistings{\listof{codelisting}{List of Listings}}
\makeatother
\makeatletter
\makeatother
\makeatletter
\@ifpackageloaded{caption}{}{\usepackage{caption}}
\@ifpackageloaded{subcaption}{}{\usepackage{subcaption}}
\makeatother
\ifLuaTeX
  \usepackage{selnolig}  % disable illegal ligatures
\fi
\usepackage{bookmark}

\IfFileExists{xurl.sty}{\usepackage{xurl}}{} % add URL line breaks if available
\urlstyle{same} % disable monospaced font for URLs
\hypersetup{
  pdftitle={TP1 (FAS-1001)},
  pdfauthor={Apolline Tritz},
  colorlinks=true,
  linkcolor={blue},
  filecolor={Maroon},
  citecolor={Blue},
  urlcolor={Blue},
  pdfcreator={LaTeX via pandoc}}

\title{TP1 (FAS-1001)}
\author{Apolline Tritz}
\date{2024-01-22}

\begin{document}
\maketitle

\begin{figure}[H]

{\centering \includegraphics{/Users/apolline/Desktop/FAS1001/fas_1001_Tritz/_tp/captureEcran.png}

}

\caption{Capture d'écran}

\end{figure}%

\section{Explication de l'image}\label{explication-de-limage}

\paragraph{1. Tout d'abord, j'ai cloné le chemin HTML du dossier, afin
de le mettre dans mon ordinateur, dans le dossier
``IntroductionMégadonnées''. J'ai utilisé git
clone.}\label{tout-dabord-jai-clonuxe9-le-chemin-html-du-dossier-afin-de-le-mettre-dans-mon-ordinateur-dans-le-dossier-introductionmuxe9gadonnuxe9es.-jai-utilisuxe9-git-clone.}

\paragraph{2. J'ai dû créer ma branche, afin de sortir de la branche
principale. Cela m'a permi d'y apporter des modifications, sans impacter
les autres. J'ai utilisé git branch et git
checkout.}\label{jai-duxfb-cruxe9er-ma-branche-afin-de-sortir-de-la-branche-principale.-cela-ma-permi-dy-apporter-des-modifications-sans-impacter-les-autres.-jai-utilisuxe9-git-branch-et-git-checkout.}

\paragraph{3. J'ai navigué dans les dossiers de mon ordinateur, afin
d'aller dans le dossier -tp, où j'avais enregistré mon fichier quarto
pour le premier TP, afin de faire mon premier commit. J'ai utilisé cd,
git add . et git commit -m
``\,``.}\label{jai-naviguuxe9-dans-les-dossiers-de-mon-ordinateur-afin-daller-dans-le-dossier--tp-ouxf9-javais-enregistruxe9-mon-fichier-quarto-pour-le-premier-tp-afin-de-faire-mon-premier-commit.-jai-utilisuxe9-cd-git-add-.-et-git-commit--m-.}

\paragraph{4. Afin de faire mon deuxième commit, j'ai dû me mettre dans
le dossier, que j'ai créé (\_travail\_session), où il y avait le
deuxième fichier quarto que j'avais produit. J'ai utilisé cd .., cd, git
add . et git commit -m
``\,``.}\label{afin-de-faire-mon-deuxiuxe8me-commit-jai-duxfb-me-mettre-dans-le-dossier-que-jai-cruxe9uxe9-_travail_session-ouxf9-il-y-avait-le-deuxiuxe8me-fichier-quarto-que-javais-produit.-jai-utilisuxe9-cd-..-cd-git-add-.-et-git-commit--m-.}

\paragraph{5. J'ai souhaité voir les deux commits que j'ai fait. J'ai
utilisé git
log}\label{jai-souhaituxe9-voir-les-deux-commits-que-jai-fait.-jai-utilisuxe9-git-log}

\paragraph{6. Je veux envoyer mes modifications sur git hub. J'ai
utilisé git push origin (nom de ma
branche).}\label{je-veux-envoyer-mes-modifications-sur-git-hub.-jai-utilisuxe9-git-push-origin-nom-de-ma-branche.}



\end{document}
